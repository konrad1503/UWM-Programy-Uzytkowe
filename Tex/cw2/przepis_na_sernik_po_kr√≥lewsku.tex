\documentclass[12pt, a4paper, titlepage]{article}
\usepackage[left=3.5cm, right=2.5cm, top=2.5cm, bottom=2.5cm]{geometry}
\usepackage[MeX]{polski}
\usepackage[utf8]{inputenc}
\usepackage{graphicx}
\usepackage{enumerate}
\usepackage{amsmath} %pakiet matematyczny
\usepackage{amssymb} %pakiet dodatkowych symboli
\title {Przepis na sernik po królewsku}
\author {Konrad Zieliński}
\date {18 październik 2022}
\begin{document}
\maketitle
\section{Składniki}
\subsection{Kakaowe ciasto kruche}
\begin{enumerate}[-]
\item 2 szklanki (320 g) mąki pszennej
\item szczypta soli
\item 1 i 1/2 łyżeczki proszku do pieczenia
\item 3 łyżki kakao
\item 3/4 szklanki cukru
\item 200 g masła (twardego)
\item 3 żółtka lub 2 jajka
\end{enumerate}
\subsection{Masa serowa*}
\begin{enumerate}[-]
\item 1 kg zmielonego trzykrotnie tłustego twarogu
\item 200 g masła (miękkiego)
\item 3/4 szklanki cukru
\item 1 opakowanie cukru wanilinowego
\item 4 jajka
\item 1 opakowanie budyniu śmietankowego
\end{enumerate}
\section{Sposób przygotowania}
\subsection{Kakaowe ciasto kruche}
\begin{enumerate}[-]
\item Prostokątną formę o wymiarach 20 x 30 cm posmarować masłem i wyłożyć papierem do pieczenia.

\item Do miski wsypać mąkę, dodać sól, proszek do pieczenia, kakao i cukier oraz pokrojone w kosteczkę zimne masło. Rozdrabniać składniki aż powstanie drobna kruszonka (palcami, mieszadłem miksera lub siekać nożem na stolnicy). Dodać jajka i połączyć składniki w gładkie oraz jednolite ciasto. Uformować kulę, podzielić na 2 części.

\item Jedną część ciasta włożyć do lodówki, drugą pokroić na plasterki i wyłożyć nimi spód formy, ugnieść palcami na równy placek. Podziurkować widelcem.
\end{enumerate}
\subsection{Masa serowa}
\begin{enumerate}[-]
\item Piekarnik nagrzać do \underline{180 stopni C}. Miękkie masło ubijać z cukrem i cukrem wanilinowym przez około \underline{5 minut} aż będzie białe i puszyste.

\item Następnie stopniowo dodawać po \underline{jednym żółtku} cały czas ubijając. Wciąż ubijając dodawać po łyżce twaróg. Na koniec zmiksować z budyniem.

\item Białka ubić w \underline{oddzielnej} misce na sztywną pianę, dodać do masy serowej i wymieszać lub zmiksować na małych obrotach miksera.

\item Masę serową wyłożyć na spód. Na wierzch zetrzeć odłożoną część ciasta (na tarce o dużych oczkach). Wstawić do piekarnika i piec przez \underline{45 minut}.

\item Wyjąć z piekarnika i ostudzić. Można posypać cukrem pudrem lub polać lukrem cytrynowym.
\end{enumerate}
\section{Wskazówki}

\textbf{* można zrobić masę serową z dodatkiem śmietanki 30\% zamiast masła, z następujących składników:}
\begin{enumerate}[-]
\item 1 kg zmielonego twarogu sernikowego
\item 200 g cukru
\item 30 g mąki ziemniaczanej
\item 1 łyżka cukru wanilinowego
\item 200 ml śmietanki 30%
\item 3 jajka
\end{enumerate}
Piekarnik nagrzać do \underline{180 stopni C}. Do misy miksera włożyć ok. 1/4 ilości sera, dodać cały cukier oraz całą mąką ziemniaczaną. Miksować na średnich obrotach miksera przez \underline{ok. 2 - 3 minuty}. Dodać resztę sera, jajka, cukier wanilinowy i znów miksować przez 1 minutę na większych obrotach na jednolitą masę. Wlać śmietankę i miksować aż wszystkie składniki dobrze się połączą.

\end{document}